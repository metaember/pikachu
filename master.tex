\documentclass[12pt,a4paper,oneside,openany]{memoir}

\usepackage[utf8]{inputenc}
\usepackage[frenchb]{babel}
\usepackage[T1]{fontenc}


% Paquets utiles
\usepackage{graphicx}
\usepackage[usenames,svgnames]{xcolor}
\usepackage{tikz}
\usepackage{asymptote}
\usepackage{flafter}
\usepackage{multirow}
\usepackage{ifdraft}
\usepackage{units}
%\usepackage[pdfpagelabels,draft,implicit=false]{hyperref}

%Tkiz
\usetikzlibrary{calc}
\usetikzlibrary{arrows}


% Décoration
\usepackage[lighttt]{lmodern} % pour le tt


% POUR MATHDESIGN : restaurer hrulefill
\let\xhrf\hrulefill
\usepackage[utopia]{mathdesign}
\let\hrulefill\xhrf
\renewcommand*{\chapterheadstart}{\begingroup
	\vspace*{\beforechapskip}%
	\begin{adjustwidth}{}{-\chapindent}%
		\hrulefill
		\smash{\rule{0.4pt}{15mm}}
	\end{adjustwidth}\endgroup}
%\usepackage{tgpagella}
\let\mathbb\relax \usepackage{bbold}
\fixpdflayout
%\counterwithout{section}{chapter}
%\setsecnumdepth{subsection}


\chapterstyle{ell}
\setsecheadstyle{\Large\bfseries\sffamily\raggedright}
\setsubsecheadstyle{\large\bfseries\sffamily\raggedright}
\midsloppy

\makepagestyle{myruled}
\makeoddfoot{myruled}{}{\thepage}{}
\makeevenfoot{myruled}{}{\thepage}{}
\makeheadrule{myruled}{\textwidth}{\normalrulethickness}
\makeevenhead{myruled}{\rightmark}{}{}
\makeoddhead{myruled}{\rightmark}{}{}
\pagestyle{myruled}

\captiondelim{ : }
\captiontitlefont{\sffamily}

\newsubfloat{figure}
\newsubfloat{table}

\newcommand{\anonchapter}[1]{\chapter*{#1}\addcontentsline{toc}{chapter}{\numberline{}#1}}
\newcommand{\todo}[1]{\marginpar{\textcolor{red}{#1}}}
\newcommand{\fltbrule}{\hrule\vspace{\onelineskip}}
\newcommand{\fltarule}{\vspace{\onelineskip}
	\hrule\vspace{\onelineskip}}

\author{Charles Bine\\Matthieu Félix}
\title{\\Production d'énergie électrique (centrales nucléaires et barrages hydroélectriques)}
\date{Décembre 2015}


% EXEMPLES & référence rapide
% dessin asymptote
% \lstinputlisting[firstline=, lastline=]{fichier}

\begin{document}
	
% Page de titre
\keepthetitle
\begin{titlingpage}
\noindent
\begin{minipage}[t]{0.4\textwidth} \begin{flushleft}
\theauthor \\ \thedate
\end{flushleft} \end{minipage}
\begin{minipage}[t]{0.4\textwidth} \begin{flushright}
Rapport de projet
Cours de réseaux électriques
\end{flushright} \end{minipage}

\vspace{3cm}
\begin{center}
{\LARGE \textsc{\thetitle}}
\ifdraft{\par Brouillon}{}
\end{center}
\vspace{3cm}

\end{titlingpage}

\addtolength{\marginparwidth}{11mm}
\abnormalparskip{4mm}

\clearpage

\tableofcontents

\clearpage


\chapter{Introduction}

Dans le sillage de la mondialisation, les enjeux de la production et de la consommation énergétiques sont capitaux. La demande énergétique mondiale, toujours croissante, requiert l'inauguration d'un nombre toujours plus important de centrales. 




\chapter{Aperçu de la production d'énergie électrique}

Les principales ressources énergétiques sont les énergies fossiles (le gaz naturel, le charbon, le pétrole), l’énergie hydroélectrique, l’énergie éolienne, l’énergie nucléaire, l’énergie solaire et l'énergie géothermique. 

Étant donné les impacts environnementaux des énergies fossiles, nous nous attarderons dans cet article à l'analyse de deux types de production d'énergie électrique renouvelable: les centrales nucléaires et les barrages hydroélectriques. 

\section{Production mondiale}

\section{Production Française}

Après la disparition complète de la production française de charbon en 2005, le pétrole, le gaz et surtout l’électricité sont les principales énergies consommées en France. Si la France ne produit plus de pétrole brut que de façon marginale, les treize raffineries implantées sur le territoire permettent de satisfaire plus de 90 \% de la demande nationale. Le groupe français Total, qui possède des concessions dans le monde entier, est la sixième entreprise mondiale et la cinquième du secteur. La part du gaz dans la consommation énergétique française a fortement augmenté depuis les années 1970, mais il s’agit à 97 \% de gaz importé, notamment de Russie, d’Algérie et de la mer du Nord. En revanche, la France produit plus d’électricité qu’elle n’en consomme, notamment grâce à 59 réacteurs nucléaires (le second parc mondial après le parc américain) qui produisaient en 2013 près de 74\% de l’électricité du pays, mais dont le bilan environnemental est l’objet de débats. Quant aux énergies renouvelables, leur part dans la production électrique française augmente et s’établit en 2013 à près de 17 \%, en grande partie grâce à l’hydroélectrique. 

La consommation française d'électricité s'élève à environ \unit[480]{TWh} par an. Cependant, la consommation est assujettie à des fluctuations parfois importantes, souvent dues à des contingences météorologiques. 

\begin{figure}[h]
	\includegraphics[width=\textwidth]{img/centrale1.jpeg}
	\caption[Une photographie de la Centrale nucléaire de Chooz]{Une photographie de la Centrale nucléaire de Chooz, construite et opérée par EDF.}
\end{figure}


\chapter{L'alternateur, au cœur de la production d'énergie électrique}

\chapter{La centrale nucléaire}

\chapter{Le barrage hydroélectrique}

\end{document}

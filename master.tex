\documentclass[12pt,a4paper,oneside,openany]{memoir}

\usepackage[utf8]{inputenc}
\usepackage[frenchb]{babel}
\usepackage[T1]{fontenc}

% Paquets utiles
\usepackage{graphicx}
\usepackage[usenames,svgnames]{xcolor}
\usepackage{tikz}
\usepackage{asymptote}
\usepackage{flafter}
\usepackage{multirow}
\usepackage{ifdraft}
\usepackage{units}
\usepackage{booktabs}
%\usepackage[pdfpagelabels,draft,implicit=false]{hyperref}

%Tkiz
\usetikzlibrary{calc}
\usetikzlibrary{arrows}


% Décoration
\usepackage[lighttt]{lmodern} % pour le tt


% POUR MATHDESIGN : restaurer hrulefill
\let\xhrf\hrulefill
\usepackage[utopia]{mathdesign}
\let\hrulefill\xhrf
\renewcommand*{\chapterheadstart}{\begingroup
	\vspace*{\beforechapskip}%
	\begin{adjustwidth}{}{-\chapindent}%
		\hrulefill
		\smash{\rule{0.4pt}{15mm}}
	\end{adjustwidth}\endgroup}
%\usepackage{tgpagella}
\let\mathbb\relax \usepackage{bbold}
\fixpdflayout
%\counterwithout{section}{chapter}
%\setsecnumdepth{subsection}


\chapterstyle{ell}
\setsecheadstyle{\Large\bfseries\sffamily\raggedright}
\setsubsecheadstyle{\large\bfseries\sffamily\raggedright}
\midsloppy

\makepagestyle{myruled}
\makeoddfoot{myruled}{}{\thepage}{}
\makeevenfoot{myruled}{}{\thepage}{}
\makeheadrule{myruled}{\textwidth}{\normalrulethickness}
\makeevenhead{myruled}{\rightmark}{}{}
\makeoddhead{myruled}{\rightmark}{}{}
\pagestyle{myruled}

\captiondelim{ : }
\captiontitlefont{\sffamily}

\newsubfloat{figure}
\newsubfloat{table}

\newcommand{\anonchapter}[1]{\chapter*{#1}\addcontentsline{toc}{chapter}{\numberline{}#1}}
\newcommand{\todo}[1]{\marginpar{\textcolor{red}{#1}}}
\newcommand{\fltbrule}{\hrule\vspace{\onelineskip}}
\newcommand{\fltarule}{\vspace{\onelineskip}
	\hrule\vspace{\onelineskip}}

\author{Charles Bine\\Matthieu Félix}
\title{\\Production d'énergie électrique (centrales nucléaires et barrages hydroélectriques)}
\date{Décembre 2015}


% EXEMPLES & référence rapide
% dessin asymptote
% \lstinputlisting[firstline=, lastline=]{fichier}

\begin{document}
	
% Page de titre
\keepthetitle
\begin{titlingpage}
\noindent
\begin{minipage}[t]{0.5\textwidth} \begin{flushleft}
\theauthor \\ \thedate
\end{flushleft} \end{minipage}
\begin{minipage}[t]{0.5\textwidth} \begin{flushright}
Rapport de projet \\
Cours de réseaux électriques
\end{flushright} \end{minipage}

\vspace{3cm}
\begin{center}
{\LARGE \textbf{\thetitle}}
\ifdraft{\par Brouillon}{}
\end{center}
\vspace{3cm}

\end{titlingpage}

\addtolength{\marginparwidth}{11mm}
\abnormalparskip{4mm}

\clearpage

\tableofcontents
\listoffigures
\listoftables
\clearpage


\chapter{Introduction}

Dans le sillage de la mondialisation, les enjeux de la production et de la consommation énergétiques sont capitaux. La demande énergétique mondiale, toujours croissante, requiert l'inauguration d'un nombre toujours plus important de centrales de production d'énergie.

Mais comment produire durablement cette énergie? Quelles méthodes choisir? Comment respecter aux mieux les recommandation établies lord de la COP 21 à Paris? 


\chapter{Aperçu de la production d'énergie électrique}

Les principales ressources énergétiques sont les énergies fossiles (le gaz naturel, le charbon, le pétrole), l’énergie hydroélectrique, l’énergie éolienne, l’énergie nucléaire, l’énergie solaire et l'énergie géothermique. 

Étant donné les impacts environnementaux des énergies fossiles, nous nous attarderons dans cet article à l'analyse de deux types de production d'énergie électrique renouvelable: les centrales nucléaires et les barrages hydroélectriques. 

\section{Production mondiale}

\section{Production Française}

Après la disparition complète de la production française de charbon en 2005, le pétrole, le gaz et surtout l’électricité sont les principales énergies consommées en France. Si la France ne produit plus de pétrole brut que de façon marginale, les treize raffineries implantées sur le territoire permettent de satisfaire plus de 90 \% de la demande nationale. Le groupe français Total, qui possède des concessions dans le monde entier, est la sixième entreprise mondiale et la cinquième du secteur. La part du gaz dans la consommation énergétique française a fortement augmenté depuis les années 1970, mais il s’agit à 97 \% de gaz importé, notamment de Russie, d’Algérie et de la mer du Nord. 

En revanche, la France produit plus d’électricité qu’elle n’en consomme, notamment grâce à 59 réacteurs nucléaires (le second parc mondial après le parc américain) qui produisaient en 2013 près de 74\% de l’électricité du pays, mais dont le bilan environnemental est l’objet de débats. Quant aux énergies renouvelables, leur part dans la production électrique française augmente et s’établit en 2013 à près de 17 \%, en grande partie grâce à l’hydroélectrique. 

Le tableau \ref{repartition production} détaille les différents composants de la production énergétique française en fonction du type de production d'énergie.

	\begin{table}[h]
		\centering
		\begin{tabular}{cc}
			\toprule
			Filière & Répartition\\
			\midrule
			Nucléaire & $74\%$ \\ 
			Hydro & $13,3\%$ \\ 
			Thermique & $7,8\%$ \\ 
			Éoliennes & $2,8\%$ \\ 
			Autres & $0,9\%$ \\ 
			Biomasse & $0,7\%$ \\ 
			\bottomrule
		\end{tabular} 
		\caption{Répartition de la production énergétique par filière}
		\label{repartition production}
	\end{table}


La consommation française d'électricité s'élève à environ \unit[480]{TWh} par an. Cependant, la consommation est assujettie à des fluctuations parfois importantes, souvent dues à des contingences météorologiques. 

\begin{figure}[h]
	\includegraphics[width=\textwidth]{img/centrale1.jpeg}
	\caption[Une photographie de la Centrale nucléaire de Chooz]{Une photographie de la Centrale nucléaire de Chooz, construite et opérée par EDF.}
\end{figure}

La répartition de la consommation de l'électricité par secteur est représentée dans la table \ref{repartition secteur}. On constate que les deux tiers de la consommation française sont dus aux ménages et aux transports.



	\begin{table}[h]
		\centering
		\begin{tabular}{cc}
			\toprule
			Secteur & Répartition\\
			\midrule
			Ménages & $30,3\%$ \\ 
			Industrie & $19,4\%$ \\ 
			Transports & $31,1\%$ \\ 
			Services & $16\%$ \\ 
			Agriculture & $3\%$ \\ 
			Pèche & $0,2\%$ \\ 
			\bottomrule
		\end{tabular} 
		\caption{Répartition de la consommation énergétique par secteur}
		\label{repartition secteur}
	\end{table}


\chapter{Le turbo-alternateur, au cœur de la production d'énergie électrique}

\begin{figure}
	\centering
	\includegraphics[width=0.95\linewidth]{Turbogenerator01}
	\caption{Un turbo-alternateur moderne}
	\label{fig:turbogenerator}
\end{figure}

\begin{figure}
	\centering
	\includegraphics[width=0.9\linewidth]{coupe_turbo_alternateur}
	\caption{Un turbo-alternateur à eau vu en coupe}
	\label{fig:coupe_turbo_alternateur}
\end{figure}

De nombreux types de centrales modernes (à gaz, à charbon, hydro-électriques, nucléaires,\textellipsis) fonctionnent à l'aide de \emph{turbo-alternateurs} : une association entre une turbine et un alternateur.

\section{Turbine}

\section{Alternateur}
L'alternateur permet de convertir un mouvement mécanique de rotation, fourni par la turbine, en un courant électrique alternatif généralement triphasé.

L'alternateur fonctionne avec un rotor et un stator. Le rotor est entraîné par la turbine et est composé d'une série d'électroaimants ; le stator est fixé au bâti, et est recouvert de bobines dans lesquelles le courant sera induit.

Le bobinage de cette partie s'effectue comme celui des machines à courant alternatif, décrit dans le cours. On notera qu'il est assez fréquent que le bobinage du stator soit fait avec plus de deux pôles, ce qui permet de réduire la vitesse de rotation en conservant la fréquence du courant produit. Ainsi, les grandes centrales hydroélectriques utilisent généralement entre 16 et 64 pôles dans leurs alternateurs.

Le fait que les électroaimants soient situés sur le rotor pose également un problème, puisqu'ils doivent être alimentés. Plutôt que d'utiliser des balais et une piste circulaire pour transmettre le courant, les grands turbo-alternateurs se composent en fait de deux alternateurs. L'alternateur secondaire est monté dans la configuration inverse, c'est à dire que les aimants sont montés sur le stator et le bobinage passif est monté sur le rotor. Le courant ainsi produit sur le rotor est redressé, puis utilisé pour alimenter les électroaimants.

On notera aussi que l'intensité du courant dans les électroaimants de l'alternateur principale peut être modifiée pour adapter le courant produit par celui-ci : un courant plus grand dans les électroaimants induira un courant produit plus grand, mais une vitesse de rotation plus faible.


\subsection{Refroidissement}
Le refroidissement des alternateurs est un problème important puisque ceux-ci tournent souvent en continu (particulièrement dans le cas des centrales nucléaires) et à de fortes charges. Le refroidissement des alternateurs se fait couramment à l'aide d'hydrogène, produit dans les centrales par électrolyse et que l'on fait circuler entre le rotor et le stator. Cet hydrogène est ensuite refroidi dans un échangeur thermique avec de l'eau.

Cette solution relativement peu onéreuse (même s'il convient alors de faire très attention à l'étanchéité de l'alternateur) permet d'obtenir de très hautes disponibilités des turbo-alternateurs.


\section{Synchronisation}


\chapter{La centrale nucléaire}


\chapter{Le barrage hydroélectrique}


\end{document}

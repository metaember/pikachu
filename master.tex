\documentclass[12pt,a4paper,oneside,openany]{memoir}

\usepackage[utf8]{inputenc}
\usepackage[frenchb]{babel}
\usepackage[T1]{fontenc}
\usepackage{fixltx2e}

% Paquets utiles
\usepackage{graphicx}
\usepackage[usenames,svgnames]{xcolor}
\usepackage{tikz}
\usepackage{asymptote}
\usepackage{flafter}
\usepackage{multirow}
\usepackage{ifdraft}
%\usepackage[pdfpagelabels,draft,implicit=false]{hyperref}

%Tkiz
\usetikzlibrary{calc}
\usetikzlibrary{arrows}


% Décoration
\usepackage[lighttt]{lmodern} % pour le tt


% POUR MATHDESIGN : restaurer hrulefill
\let\xhrf\hrulefill
\usepackage[utopia]{mathdesign}
\let\hrulefill\xhrf
\renewcommand*{\chapterheadstart}{\begingroup
	\vspace*{\beforechapskip}%
	\begin{adjustwidth}{}{-\chapindent}%
		\hrulefill
		\smash{\rule{0.4pt}{15mm}}
	\end{adjustwidth}\endgroup}
%\usepackage{tgpagella}
\let\mathbb\relax \usepackage{bbold}
\fixpdflayout
%\counterwithout{section}{chapter}
%\setsecnumdepth{subsection}


\chapterstyle{ell}
\setsecheadstyle{\Large\bfseries\sffamily\raggedright}
\setsubsecheadstyle{\large\bfseries\sffamily\raggedright}
\midsloppy

\makepagestyle{myruled}
\makeoddfoot{myruled}{}{\thepage}{}
\makeevenfoot{myruled}{}{\thepage}{}
\makeheadrule{myruled}{\textwidth}{\normalrulethickness}
\makeevenhead{myruled}{\rightmark}{}{}
\makeoddhead{myruled}{\rightmark}{}{}
\pagestyle{myruled}

\captiondelim{ : }
\captiontitlefont{\sffamily}

\newsubfloat{figure}
\newsubfloat{table}

\newcommand{\anonchapter}[1]{\chapter*{#1}\addcontentsline{toc}{chapter}{\numberline{}#1}}
\newcommand{\todo}[1]{\marginpar{\textcolor{red}{#1}}}
\newcommand{\fltbrule}{\hrule\vspace{\onelineskip}}
\newcommand{\fltarule}{\vspace{\onelineskip}
	\hrule\vspace{\onelineskip}}

\author{Charles Bine\\Matthieu Félix}
\title{\\Production d'énergie électrique (centrales nucléaires et barrages hydroélectriques)}
\date{Décembre 2015}


% EXEMPLES & référence rapide
% dessin asymptote
% \lstinputlisting[firstline=, lastline=]{fichier}

\begin{document}
	
% Page de titre
\keepthetitle
\begin{titlingpage}
\noindent
\begin{minipage}[t]{0.4\textwidth} \begin{flushleft}
\theauthor \\ \thedate
\end{flushleft} \end{minipage}
\begin{minipage}[t]{0.4\textwidth} \begin{flushright}
Rapport de projet
Cours de réseaux électriques
\end{flushright} \end{minipage}

\vspace{3cm}
\begin{center}
{\LARGE \textsc{\thetitle}}
\ifdraft{\par Brouillon}{}
\end{center}
\vspace{3cm}

\end{titlingpage}

\addtolength{\marginparwidth}{11mm}
\abnormalparskip{4mm}

\clearpage

\tableofcontents

\clearpage


\chapter{Introduction}

Dans le sillage de la mondialisation, les enjeux de la production et de la consommation énergétiques sont capitaux. La demande énergétique mondiale, toujours croissante, requiert l'inauguration d'un nombre toujours plus important de centrales. 

Les principales ressources énergétiques sont les énergies fossiles (le gaz naturel, le charbon, le pétrole), l’énergie hydroélectrique, l’énergie éolienne, l’énergie nucléaire, l’énergie solaire et l'énergie géothermique. 

Étant donné les impacts environnementaux des énergies fossiles, nous nous attarderons dans cet article à l'analyse de deux types de production d'énergie électrique renouvelable: les centrales nucléaires et les barrages hydroélectriques. 


\chapter{Aperçu de la production d'énergie électrique}




\chapter{L'alternateur, au cœur de la production d'énergie électrique}

\chapter{La centrale nucléaire}

\chapter{Le barrage hydroélectrique}

\end{document}
